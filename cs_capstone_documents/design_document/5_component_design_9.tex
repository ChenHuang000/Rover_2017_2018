\subsection{Logging / Recording Coordinator}
\subsubsection{Overview}
Recording and logging information is very important for monitoring the rover's status and re-run how a rover was performing during some set period.

\subsubsection{Design Concerns}
\begin{itemize}
\item \textit{Storage Space:} With all the video/recording the system is going to do, the space might be limited on the system or on the rover.
\item \textit{Video Bitrate:} The video should be recorded in a quality that is independent of the streaming video stream to allow full video quality when retrieved
\item \textit{Storage Format:} With the variable size of log files, some settings might be necessary to control logging location, size, or type. 
\end{itemize}

\subsubsection{Design Elements}
\begin{itemize}
\item The coordinator will take any information from the rover as a stream of information.
\item The coordinator will store any video stream that the rover sends back to save back on overhead on the rover.
\item The rover will not be saving any files to allow for overhead and processing.
\end{itemize}

\subsubsection{Design Rationale}
The rover is going to be overloaded with calculations and trans-coding, so the ground station should be able to offload any extra computation onto it.
The ground control system will take care of storing information like log files in session logs.
There is an inbuilt system using ROS that makes a bag that can be used to log everything.

