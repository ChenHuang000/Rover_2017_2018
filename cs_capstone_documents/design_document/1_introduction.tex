\subsection{Purpose}
The purpose of this software design document is to cover the design for the Oregon State University Mars Rover Team's ground station software.
It will cover the details about how the software will function, how we will implement that functionality, and the reasoning behind design decision choices that were made.
The ground station software for the Mars Rover project is the single point of contact between the team-built competition Rover and the users operating or viewing the rover.


\subsection{Scope}
This project will consist of a software package running from a remote control base station in order to remotely operate a competition-ready robot built by the Oregon State Robotics Club's Mars Rover team.
The project must be completed by May 31st 2018 in order to be ready for the University Rover Challenge taking place in Hanksville, Utah.
The controlling software must be able to do at minimum, the following: 

\begin{itemize}
\item Provide the capability to remotely drive the Rover via joysticks connected to the ground station computer.
\item Allow for viewing of up to three video feeds from the Rover, and be able to change what video streams are viewed.
\item Via a user input device, allow for manipulation of the Rover arm.
\item Provide visual feedback about the state of the joint positions of the arm as it is being moved.
\item Show the user a map of the competition area, allowing the user to zoom, view the Rover's location, and place navigation and landmark way-points.
\item Allow the user to add way-points and then place the Rover in autonomous mode, wherein the Rover will drive under its own control along the way-points provided.
\item Upon completion of an autonomous path, provide an easy to read notification that the navigation is complete.
\item Provide a myriad of status information about the Rover's current condition including, but not limited to, the following:
  \begin{itemize}
  \item Connection statuses
  \item Rover Sub-system connection statuses
  \item On-board sensor readings
  \item Battery charge level
  \item System Errors
  \end{itemize}
\item Include documentation allowing for easy reuse of the software for future Rover competition years.
\end{itemize}

Additionally, the client has requested a document be written providing a starting guide on how to re-use the finalized software package for future Rover teams to ease development in future years.


\subsection{Context}
Each year the Mars Rover software team writes ground station software from scratch in order to meet the changing requirements of both the Rover itself and competition rule changes.
As this software is the main point of contact between the users and the Rover, the success of the team during competition often hinges on how well this software performs.
In the past, lack of modularity and abstraction has made re-use of the ground station code near impossible.


\subsection{Summary}
This rest of this document will provide the details about how our team will implement the ground station software for the Mars Rover team.
We will start by covering design considerations such as the environment our software will be running in, as well as the specific concerns of our stakeholder. 
Immediately following will be a breakdown of the individual software components we will be writing.
These will include descriptions of what, how, and why we will be writing each component the way we are.
The document ends with a visual breakdown of the software UI, and rationale for why each component looks the way it does.